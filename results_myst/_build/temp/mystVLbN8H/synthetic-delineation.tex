% Created with jtex v.1.0.17
\documentclass{article}
\PassOptionsToPackage{short, nodayofweek}{datetime}

\input{curvenote.def}



% colors for hyperlinks
\hypersetup{colorlinks=true, allcolors=blue}
\hypersetup{
pdftitle={\@title},
pdfsubject={},
pdfauthor={\@author},
pdfkeywords={},
addtopdfcreator={Written in Curvenote}
}

\usepackage{curvenote}

\title{Synthetic irrigation model}

\newdate{articleDate}{21}{4}{2024}
\date{\displaydate{articleDate}}

\author{\bfseries Hector Nieto\mdseries\\ICA-CSIC\\\AND\bfseries Benjamin Mary\mdseries\\ICA-CSIC\\\AND\bfseries Vicente Burchard-Levine\mdseries\\ICA-CSIC\\}

\begin{document}

\maketitle
\keywords{DA}

\section{Abstract}

\begin{itemize}
\item \textbf{Baseline ET} = ET without irrigation scheme from a land-surface hydrogeological model
\end{itemize}

\section{Background and objectives}

In this study, to compute the \textbf{baseline ET} (ET without irrigation scheme), we use the CATHY code (Catchment Hydrology, Camporese et al., 2010) for modeling soil surface-subsurface water flow. We conducted experiments involving synthetic scenarios to assess their detectability with:

\begin{itemize}
\item varying sizes and levels of irrigation
\item different frequencies of evapotranspiration (EO)
\item atmospheric boundary conditions, including rain events occurring between irrigation cycles.
\end{itemize}

\section{Description of the synthetic models}

\subsection{Hydrological model inputs}

The \textbf{regional domain size} is 100x100m square, flat. \textbf{Irrigation local size} is a 30x30m square centered on the regional domain.
The vegetation is uniform on all the regional domain size, with \textbf{root depth} of 1m typical from herbaceous crop.
The soil is homogeneous all over the regional domain.

The irrigation consist in a single event at ?? sec (day ??).
The irrigation rate is \textbf{5e-07} m/s during \textbf{21600} sec. This is equivalent to 43.2 mm/day during 6hours.

The \textbf{potential ETp} is homogeneous all over the domain and with time, set to -1e-07 m/s.

\textbf{At time 0}, the soil is dry with an initial pressure head of -30m that is equivalent to a saturation level of 0.3?.

\textbf{No flow} boundary conditions are imposed outside the regional domain.

\subsection{Earth observations}

Earth Observations are available at a \textbf{daily frequency}. To mimick EO we used a hydrological model including the irrigation.

\begin{figure}[!htbp]
\centering
\includegraphics[width=0.75\linewidth]{files/EOMAJI_mesh-509781dd4b086cb32383299dabc9f88d.png}
\caption*{EO-MAJI-IrrDelineation}
\end{figure}

\begin{figure}[!htbp]
\centering
\includegraphics[width=0.75\linewidth]{files/vtksaturation-6351a4b146f09c4ae6b6fcd05adccf86.gif}
\caption*{Saturation over the course of the irrigation}
\end{figure}

\bigskip\noindent
\begin{tabular}{p{\dimexpr 0.333\linewidth-2\tabcolsep}p{\dimexpr 0.333\linewidth-2\tabcolsep}p{\dimexpr 0.333\linewidth-2\tabcolsep}}
\toprule
Key & S0 & S1 \\
\hline
ETp & -1e-07 &  \\
nb\_days & 10 &  \\
nb\_hours\_ET & 8 &  \\
irr\_time\_index & 3 &  \\
irr\_length & 21600 &  \\
irr\_flow & 5e-07 &  \\
EO\_freq\_days & 1 &  \\
ETp\_window\_size\_x & 10 &  \\
\bottomrule
\end{tabular}

\bigskip
\end{document}
